\documentclass[11pt,a4paper]{article}
\usepackage[utf8]{inputenc}
\usepackage[english]{babel}
\usepackage{amsmath}
\usepackage{amsfonts}
\usepackage{amssymb}
\usepackage{graphicx, url,hyperref,booktabs}
\usepackage[left=2cm,right=2cm,top=2cm,bottom=2cm]{geometry}
\author{Claudius Gr\"abner}
\begin{document}
\section{Model description}
We first build a very primitive and simple model for the problem at hand.
We then add more complex decision making rules for the agents. 
This allows us the isolate the systemic effects of the changes in individual decision making.

All implementations of the model follow the same sequence of events, as illustrated in table \ref{tab:procedure}.
Agents first make a decision which seed to use. The decision making procedure depends on the model specification.
At the end of the time step, agents receive their payoff according to the procedure mandated by the particular model specification.\footnote{We will keep the model description short. 
The source code is freely available at \href{https://github.com/graebnerc/New-McDonald}{https://github.com/graebnerc/New-McDonald}.}

\begin{table}
\begin{center}
\begin{tabular}{ll}
\toprule
(0.) & (Only in $t_0$: Initialization of the population and spatial allocation of the farmers.)\\\midrule
1. & Calculation of the mean yield for the NP farmers (not in $t_0$).\\\midrule
2. & Farmers choose their seed type sequentially (in a randomly determined sequence).\\\midrule
3. & Recording of all relevant state variables.\\\bottomrule
\end{tabular}
\caption{The sequence of events in the simulation.}\label{tab:procedure}
\end{center}
\end{table}


\subsection{Baseline model}
We have a number of $N$ farmers, each of which needs to choose between the proprietary and non-proprietary seed. 
In the baseline model, agents choose the seed randomly according to a probability fixed as a parameter.
At the end of each time step, agents receive their payoffs.
The yield of the P seed is fixed (and known to every agent) at $Y_P$.
The yield of the NP seed is stochastic at the individual level following a truncated normal distribution with mean $Y_P$ and given variance $\sigma^2$ over the positive reals.

\subsection{Model extension 1}
In the first extension we relax the assumption of random choice of seed.
Rather, agents compare the yield of the two seed types in previous time steps and pick those with the higher value.
There are basically three different implementations of this and we explore all of them in our model:
\begin{enumerate}
\item  The agents compare the fixed return of the proprietary seed with the average yield of all agents that have used the non-proprietary alternative in the last k rounds (k is a parameter).
\item The agents are located on a grid (with a von Neumann neighborhood) and compare the P payoff with the average payoffs of their neighbors using the NP seed (again considering the previous k rounds).
\item This case does not differ to case B, but agents put more weight on their own payoff in the previous round (with a 50 per cent) weight.
\end{enumerate}
In any case, we introduce a parameter $k$ which specifies the memory of the farmers (i.e. the number of previous time steps they consider for the calculation of past yields.\footnote{This means that if $k=1$ farmers only compare the yields of the previous round. If $k=10$ they compare the average yield over the past 10 periods.}

\subsection{Model extension 2}
The second extension allows for network effects of the NP seed. 
The expected yield is now a function of the number of agents using this seed.
If $Y_P$ and $Y_{NP}$ denotes the yield of the P and NP seed respectively, $n$ is the number of farmers currently using the NP seed, and $N$ the total number of farmers, we have the following expression:
\begin{equation}
Y_{NP} =  Y_P \frac{n}{N}
\end{equation}
The rationale behind this is the increasing potential for innovation in such an ``open source'' environment.\footnote{In future work, this mechanism could be modeled explicitly using a genetic algorithm searching the space of possible seeds.}

\end{document}